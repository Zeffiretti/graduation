%%
% The BIThesis Template for Bachelor Graduation Thesis
%
% 北京理工大学毕业设计(论文)致谢 —— 使用 XeLaTeX 编译
%
% Copyright 2020 Spencer Woo
%
% This work may be distributed and/or modified under the
% conditions of the LaTeX Project Public License, either version 1.3
% of this license or (at your option) any later version.
% The latest version of this license is in
%   http://www.latex-project.org/lppl.txt
% and version 1.3 or later is part of all distributions of LaTeX
% version 2005/12/01 or later.
%
% This work has the LPPL maintenance status `maintained'.
%
% The Current Maintainer of this work is Spencer Woo.
%
% Compile with: xelatex -> biber -> xelatex -> xelatex

\unnumchapter{致~~~~谢}
\renewcommand{\thechapter}{致谢}

\ctexset{
  section/number = \arabic{section}
}

% 致谢部分尽量不使用 \subsection 二级标题,只使用 \section 一级标题

值此论文完成之际,首先向我的导师石青老师表达衷心的感激。自毕设选题以来,石青老师悉心指导,耐心释疑。不仅为我毕业设计研究的方向指明了目标,同时为研究过程中所遇到的困难提出解决方案,保持着每周至少一次的指导,督促我完成了本科毕设。石老师对待科学案件研究求真的态度和追求卓越的精神激励着我不断克服困难,解决问题。

感谢实验室的李昌、高子航和贾广禄师兄,他们为本研究的资料搜集、系统设计、数据来源收集和处理方式等提出了大量建议。尤其在新冠肺炎疫情爆发后,原定毕设课题有所调整,他们为我提供了大量帮助,使我能快速熟悉研究中所用到的软硬件设备、算法理论及实践等等。

感谢我家人的理解和支持,为我创造了宽松、自由的研究氛围。
