% \section{课题研究背景与意义}
仿生机器人是机器人研究的一大分支,旨在通过研究生物系统的结构、性状、原理、行为以及相互作用,从而为机器人设计、控制和决策提供新的设计思想、工作原理和系统构成\cite{sunFangShengXueDeXianZhuangHeWeiLai2007},是一门集生命科学、物质科学、数学与力学、信息科学、工程技术以及系统科学等学科的交叉学科\cite{wangFangShengJiQiRenYanJiuXianZhuangYuFaZhanQuShi2015}。生物体结构合理,运动灵活,具有良好的环境适应特性和良好的生存能力,仿生机器人因其部分模仿生物结构和性状特点而继承了一部分这些特性,受到研究者的青睐\cite{shenFangShengJiQiRenYanJiuJinZhanJiFangShengJiGouYanJiu2015}。生物鼠是被广泛使用的实验动物之一\cite{kongShiYanDongWuPinXiShuJuKuDeJianLi2015},对其行为模式的研究收到生物学家的广泛关注,但由于生物鼠行为随机、难以预测,相关的实验开展存在困难。作为仿生机器人的一个基本实例,设计精巧的仿生机器鼠可以模拟生物鼠各个关节的运动,从而产生与生物鼠相似的基本行为,并以此引发生物鼠的特异性反应\cite{gaoOverviewBiomimeticRobots2019}。因此,利用仿生机器鼠与生物鼠进行行为交互,探究交互过程中生物鼠的反应与仿生机器鼠行为的联系,对研究生物鼠的行为模式和机器人的控制策略均有重要意义\cite{frohnwieserUsingRobotsUnderstand2016}。

但在实践中,仿生机器鼠与生物鼠的行为交互仍然面临诸多困难。首先,两者交互时行为的一致性有赖于双方,特别是仿生机器鼠反应的快速性\cite{kleinRobotsServiceAnimal2012}。而生物鼠身体构造精巧,动作灵活,反应机敏,模仿其身体特点设计的仿生机器鼠往往具有复杂的结构和较多的自由度,给仿生机器鼠的动作规划带来了挑战。在产生仿鼠动作过程中,仿生机器鼠的控制系统往往需要耗费大量时间计算正、逆运动学问题,使得仿生机器鼠反应迟钝,难以满足交互实验的需求。其次,生物鼠行为表现具有普遍的个体差异\cite{BarnettSTheRat},这导致难以用特定的方法对其行为加以预测或控制,给仿生机器鼠的行为生成方式提出了挑战。单一机制的行为生成算法在经过调试后,往往只能适用于某一特定的生物鼠,当切换交互对象后,相应的算法往往需要进行调整,实验的可重复性大大降低。最后,生物鼠在实验环境中往往表现出渐进的环境适应性,这意味着同一生物鼠在实验之初和实验进行一段时间后表现出的行为模式存在较明显的差异。而单一的仿生机器鼠行为生成机制无法适应生物鼠的这一行为变化,这导致了部分控制策略在初期表现优秀,但随着时间推移,其可用性往往大不如前。上述现实困难使得现有的机器鼠与生物鼠的交互实验存在着可重复性差、应用场景单一和持续时间较短等不足。

% 随着人工智能领域相关技术的发展,研究者开始关注其在仿生机器鼠与生物鼠行为交互领域的作用。典型应用包括利用深度神经网络识别生物鼠的行为模式,为仿生机器鼠的行为决策提供依据,这一应用使得仿生机器鼠具有了处理大量交互数据的能力\cite{dechaumontComputerizedVideoAnalysis2012a},但仍未对其行为生成机制产生重大影响。强化学习是机器学习中的一个领域,强调如何基于环境而行动,以取得最大化的预期利益\cite{liuShenDuQiangHuaXueXiZongShu2018}。其灵感来源于心理学中的行为主义理论,即有机体如何在环境给予的奖励或惩罚的刺激下,逐步形成对刺激的预期,产生能获得最大利益的习惯性行为。近来,基于强化学习控制的机器人在Atari游戏、围棋和德州扑克等领域表现优异,其中DeepMind的AlphaGo机器人击败围棋冠军李世石成为大众关注的热点\cite{silverMasteringGameGo2017, botvinickReinforcementLearningFast2019}。考虑到仿生机器鼠与生物鼠交互过程与Atari游戏等具有高度的相似性,利用强化学习训练机器人完成复杂环境中的任务受到人工智能和机器人领域研究者的重视。